\documentclass[letterpaper,12pt,fleqn]{article}
\usepackage[margin=64pt]{geometry}
\usepackage{amsthm}
\usepackage{amsmath}
\usepackage{amssymb}
\usepackage{parskip}
\usepackage{graphicx}
\usepackage{enumerate}

%\newcommand{\transpose}{^{\intercal}}
\newcommand{\transpose}{^{\top}}

\theoremstyle{definition}
\newtheorem{exercise}{Exercice}


\begin{document}
\pagestyle{empty}

\hrule \vspace{0.5em}
\noindent {\bf CFRM 460: Mathematical Methods for Quantitative Finance} \hfill Homework 3 \newline \hrule

\vspace{1em}

\begin{enumerate}
\item The Black-Scholes price for a European put option is
\begin{equation}
P(S, t, K, T, r, q, \sigma) = Ke^{-r(T - t)} \Phi(-d_{-}) - Se^{-q(T - t)} \Phi(-d_{+})
\label{eqn:bsp}
\end{equation}
where
$$d_{+} = \frac{\log\left(\frac{S}{K}\right) + \left(r - q + \frac{\sigma^{2}}{2} \right) \big( T - t \big)}{\sigma \sqrt{T - t}}$$
and
$$d_{-} = d_{+} - \sigma \sqrt{T - t} = \frac{\log\left(\frac{S}{K}\right) + \left(r - q - \frac{\sigma^{2}}{2} \right) \big( T - t \big)}{\sigma \sqrt{T - t}}$$

\vspace{1em}

Compute each of
\begin{enumerate}[(a)]
\item $\displaystyle \Delta(P) = \frac{\partial P}{\partial S}$
\item $\displaystyle \Gamma(P) = \frac{\partial^{2} P}{\partial S^{2}}$
\item $\displaystyle \theta(P) = \frac{\partial P}{\partial t}$
\item $\displaystyle \rho(P) = \frac{\partial P}{\partial r}$
\end{enumerate}

\vspace{1em}

by taking derivatives of (\ref{eqn:bsp}). Verify that your answers are correct using put-call parity.

\vspace{2em}

You can find expressions for \textit{The Greeks} on pages 92 and 93 of the Stefanica text.  Verify that your answer matches the expression for the put option and that put-call parity gives the expression for the call option.  And as always, verify your calculations with Mathematica.



\newpage
\textbf{Example}

Compute the vega of a European put option.
\begin{align*}
\mbox{vega}(P) = \frac{\partial P}{\partial \sigma} &= \frac{\partial}{\partial \sigma} \big[ K e^{-r(T - t)} \Phi(-d_{-}) - Se^{-q(T - t)} \Phi(-d_{+}) \big] \\
&= K e^{-r(T - t)} \phi(-d_{-}) \frac{\partial}{\partial \sigma}(-d_{-}) - Se^{-q(T - t)} \phi(-d_{+}) \frac{\partial}{\partial \sigma}(-d_{+}) \\
&= K e^{-r(T - t)} \phi(d_{-}) \frac{\partial}{\partial \sigma}(-d_{-}) - Se^{-q(T - t)} \phi(d_{+}) \frac{\partial}{\partial \sigma}(-d_{+}) \\
\end{align*}

Lemma 3.15 states that $\; K e^{-r(T - t)} \phi(d_{-}) = S e^{-q(T - t)} \phi(d_{+})$, thus

\begin{align*}
\mbox{vega}(P) = \frac{\partial P}{\partial \sigma} &= S e^{-q(T - t)} \phi(d_{+}) \frac{\partial}{\partial \sigma}(-d_{-}) - S e^{-q(T - t)} \phi(d_{+}) \frac{\partial}{\partial \sigma}(-d_{+}) \\
&= S e^{-q(T - t)} \phi(d_{+}) \left[ \frac{\partial}{\partial \sigma}(-d_{-}) -  \frac{\partial}{\partial \sigma}(-d_{+}) \right] \\
&= S e^{-q(T - t)} \phi(d_{+}) \left[ \frac{\partial}{\partial \sigma}(d_{+}) -  \frac{\partial}{\partial \sigma}(d_{-}) \right] \\
&= S e^{-q(T - t)} \phi(d_{+}) \; \frac{\partial}{\partial \sigma} \bigg[ d_{+} -  d_{-} \bigg] \\
\end{align*}

But $\; d_{-} = d_{+} - \sigma \sqrt{T - t}$ $\implies$ $d_{+} - d_{-} = \sigma \sqrt{T - t}$, thus

\begin{align*}
\mbox{vega}(P) = \frac{\partial P}{\partial \sigma} &= S e^{-q(T - t)} \phi(d_{+}) \; \frac{\partial}{\partial \sigma} \bigg[ \sigma \sqrt{T - t} \bigg] \\
\mbox{vega}(P) = \frac{\partial P}{\partial \sigma} &= S e^{-q(T - t)} \phi(d_{+}) \; \sqrt{T - t}
\end{align*}

\vspace{2em}

Check the result using put-call parity:
\begin{align*}
P &= C - Se^{-q(T - t)} + Ke^{-r(T - t)} \\
\mbox{vega}(P) = \frac{\partial P}{\partial \sigma} &= \frac{\partial}{\partial \sigma} \big[ C - Se^{-q(T - t)} + Ke^{-r(T - t)} \big] \\
&= \frac{\partial C}{\partial \sigma}
\end{align*}

The vega of a European put option is the same as the vega for a European call option.




\end{enumerate}

\end{document}
