\documentclass[letterpaper,10pt]{article}

\usepackage[margin=64pt]{geometry}
\usepackage{amsthm}
\usepackage{amsmath}
\usepackage{amssymb}
\usepackage{enumerate}
\usepackage{graphicx}
\usepackage{hyperref}
\usepackage{parskip}
\usepackage{pgfplots}
\usepackage{relsize}
\usepackage{tikz}

\usepgfplotslibrary{fillbetween}
\setcounter{secnumdepth}{0}
\allowdisplaybreaks[1]

\newcommand{\doubleu}[1]{\underline{\underline{#1}}}
\newcommand{\ddx}[1]{\frac{d}{dx} \bigg( #1 \bigg)}
\newcommand{\partiald}[2]{\frac{\partial #1}{\partial #2}}
\newcommand{\matrixpartiald}[2]{\dfrac{\partial #1}{\partial #2}}
\newcommand{\mathematica}[1]{Mathematica: \texttt{#1}}
\newcommand{\intbyparts}[7]{
	\begin{aligned}
		\text{Let}\   #1 &= #4      &      &\text{and}\    &   #2 &= #6 \\
		\Rightarrow \frac{d#1}{d#3} &= #5 \   &  &   &  \Rightarrow \frac{d#2}{d#3} &= #7
	\end{aligned}
	\\ \text{We know} \  \int #1 \, d#2 = #1#2 - \int #2 \, d#1 \\
	= {#4 \cdot #6} - \int {#6 \cdot #5} \, d#3
}
\newcommand{\threematrix}[9]{
	\begin{bmatrix}
		#1 & #4 & #7 \\
		#2 & #5 & #8 \\
		#3 & #6 & #9
	\end{bmatrix}
}
\newcommand{\threevector}[3]{
	\begin{bmatrix}
		#1 \\ #2 \\ #3
	\end{bmatrix}
}


\begin{document}
	\noindent {\bf Rukmal Weerawarana (1337197)} \newline
	\noindent {\bf CFRM 460} \newline
	\noindent {\bf Homework 5 Solutions} \newline
	\noindent {\bf 2/19/16}
	\newline \hrule

	\section{Question 1}
		\begin{gather*}
			\text{Let } A = \threematrix{1}{1}{-2}{1}{4}{1}{0}{1}{1} \\
		\end{gather*}

		\subsection{Part (a)}
			\begin{gather*}
				\text{Let } L_1 = \threematrix{1}{-1}{0}{0}{1}{0}{0}{0}{1} \\
				\Rightarrow L_1A = \threematrix{1}{-1}{0}{0}{1}{0}{0}{0}{1} \threematrix{1}{1}{-2}{1}{4}{1}{0}{1}{1} = \threematrix {1}{0}{-2}{1}{3}{1}{0}{1}{1} \\
				\text{Let } L_2 = \threematrix{1}{0}{2}{0}{1}{0}{0}{0}{1} \\
				\Rightarrow L_2 L_1 A = \threematrix{1}{0}{2}{0}{1}{0}{0}{0}{1} \threematrix {1}{0}{-2}{1}{3}{1}{0}{1}{1} = \threematrix{1}{0}{0}{1}{3}{3}{0}{1}{1} \\
				\text{Let } L_3 = \threematrix{1}{0}{0}{0}{1}{-1}{0}{0}{1} \\
				\Rightarrow L_3 L_2 L_1 A = \threematrix{1}{0}{0}{0}{1}{-1}{0}{0}{1} \threematrix{1}{0}{0}{1}{3}{3}{0}{1}{1} = \threematrix{1}{0}{0}{1}{3}{0}{0}{1}{0} \\
				\text{As the reduced matrix is in Row Echelon form, the number of pivots of A is equivalent} \\ \text{to the number of pivots in the reduced Upper-Right triangular form seen above.} \\
				\Rightarrow \text{Number of pivots in } A = 2
			\end{gather*}

		\subsection{Part (b), (c), (d) \& (e)}
			As the square matrix $A$ does not have pivots equal to its dimensions (3), it cannot have an inverse (i.e. matrix A is singular). Thus, the equation $Ax = b$ cannot have any solutions, regardless of the value of $b$.


	\section{Question 2}
		\begin{gather*}
			AB = I \\
			CA = I
		\end{gather*}
		\subsection{Part (a)}
			\begin{gather*}
				A: n \times  n \\
				B: n \times n \\
				C: n \times n
			\end{gather*}

		\subsection{Part (b)}
			\begin{gather*}
				\Rightarrow AB = CA \\
				\text{Multiplying both sides by } B \text{:} \\
				ABB = CAB \Rightarrow (AB)B = C(AB) \\
				\text{We know } AB = I \Rightarrow IB = CI \\
				\text{But, } IX = X \text{ and } XI = X \\
				\therefore B = C
			\end{gather*}

		\subsection{Part (c)}
			\begin{gather*}
				\text{We know } AB = I \text{, } CA = I \text{ and } B = C\\
				\text{Using the property that } A^{-1}A = I \text{ and } AA^{-1} = I \text{,} \\
				\text{we can conclude } B = C = A^{-1} \\
				\therefore A \text{ is invertible.}
			\end{gather*}


	\section{Question 3}
		\subsection{Part (a)}
			\begin{gather*}
				(I-A)^2 = I^2-2IA+A^2 \\
				\text{But, we know } A^2 = A \text{ and } I^n = I \\
				\Rightarrow I^2 -2IA + A^2 = I - 2A + A = \doubleu{I-A}
			\end{gather*}

		\subsection{Part (b)}
			\begin{gather*}
				(I-A)^7 = (I-A)\left[(I-A)^2\right]^3 \\
				\text{Recall, } (I-A)^2 = I-A \\
				\Rightarrow (I-A)[(I-A)^2]^3 = (I-A)(I-A)^3 = (I-A)^4 = [(I-A)^2]^2 \\
				\Rightarrow [(I-A)^2]^2 = (I-A)^2 = \doubleu{I-A}
			\end{gather*}


	\section{Question 4}
		\subsection{Part (a)}
			\begin{gather*}
				\Rightarrow a \threevector{1}{2}{3} + b \threevector{6}{4}{2} = \threevector{9}{2}{-5} \\
				\text{Multiplying both sides by a constant $c$, we have:} \\
				ac \threevector{1}{2}{3} + bc \threevector{6}{4}{2} = c \threevector{9}{2}{-5} \Rightarrow ac \threevector{1}{2}{3} + bc \threevector{6}{4}{2} - c \threevector{9}{2}{-5} = 0 \\
				\text{Reversing the dot product, we have: } \threematrix{1}{2}{3}{6}{4}{2}{9}{2}{-5} \threevector{ac}{bc}{-c} = \threevector{0}{0}{0} \\
				\text{Using Gaussian elimination to determine solutions:} \\
				\left[
				\begin{array}{ccc|c}
					1 & 6 & 9 & 0 \\
					2 & 4 & 2 & 0 \\
					3 & 2 & -5 & 0
				\end{array}
				\right] \\
				R_2 \rightarrow \frac{r_2}{2}
				\left[
				\begin{array}{ccc|c}
					1 & 6 & 9 & 0 \\
					1 & 2 & 1 & 0 \\
					3 & 2 & -5 & 0
				\end{array}
				\right] \\
				R_3 \rightarrow r_3-r_2
				\left[
				\begin{array}{ccc|c}
					1 & 6 & 9 & 0 \\
					1 & 2 & 1 & 0 \\
					2 & 0 & -6 & 0
				\end{array}
				\right] \\
				R_1 \rightarrow r_1 - 3r_2
				\left[
				\begin{array}{ccc|c}
					-2 & 0 & 6 & 0 \\
					1 & 2 & 1 & 0 \\
					2 & 0 & -6 & 0
				\end{array}
				\right] \\
				R_3 \rightarrow r_3 + r_2
				\left[
				\begin{array}{ccc|c}
					-2 & 0 & 6 & 0 \\
					1 & 2 & 1 & 0 \\
					0 & 0 & 0 & 0
				\end{array}
				\right] \\
				R_2 \rightarrow r_2 + \frac{r_1}{2}
				\left[
				\begin{array}{ccc|c}
					-2 & 0 & 6 & 0 \\
					0 & 2 & 4 & 0 \\
					0 & 0 & 0 & 0
				\end{array}
				\right] \\
				\text{Thus, as all of the coefficients are 0 as per the solution to the equation,} \\
				\text{the linear expression of the vector cannot be done.} \\
			\end{gather*}
		\subsection{Part (b)}
			\begin{gather*}
				\text{The Gaussian elimination performed above shows that}
				\threematrix{1}{2}{3}{6}{4}{2}{9}{2}{-5}
				\text{has 2 pivots.}
			\end{gather*}
\end{document}
